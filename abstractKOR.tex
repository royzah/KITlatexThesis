\begin{center}      
	\vspace*{0.5 cm}
	\begin{spacing}{1.6}
		{\fontsize{22}{30}\selectfont 대규모 산업용 IoT 환경에서의 생체모방 혼잡 제어 기법}
	\end{spacing}
	\vspace{1 cm}
	{\fontsize{14}{20}\selectfont Muhammad Royyan}
	
	\vspace{1 cm}
	{\fontsize{14}{20}\selectfont 금오공과대학교 대학원 IT융복합공학과}
	
	\vspace{1.5 cm}
	{\fontsize{14}{20}\selectfont 요약}
	
	\vspace{1 cm}
\end{center}


본 논문에서는 대규모 IIoT(Industrial IoT) 환경에서 혼잡 제어를 위한 복합 생체 모방 알고리즘을 제안한다. 네트워크의 혼잡은 한정된 자원과 센서 노드의 수에 의해 발생되며 처리량 및 end-to-end 지연에 관련된 QoS(Quality of Service)를 저하시킬 수 있다. 제안하는 알고리즘은 이러한 네트워크 혼잡을 피하기 위해 다음과 같은 방식을 사용한다. 첫째, C-LV(Competitive Lotka-Volterra) 모델을 사용하여 각 트래픽의 속도를 조절하며, 노드 간의 공정성을 유지한다. 하지만 C-LV의 매개 변수의 최적화가 요구된다. 둘째, DA(Dragonfly Algorithm)은 end-to-end 지연을 최소화하기 위한 변수를 최적화하여 C-LV의 성능을 향상시킨다. DA는 새로 설치되거나 고장 노드로 인해 노드의 개수가 변경되면 제안하는 기법을 변경하고, 최적화된 변수를 얻을 수 있도록 한다. 시뮬레이션 결과들은 제안된 기법이 IIoT 환경에서 QoS를 향상시키고, 플로우 간에 변화하는 트래픽 부하, 확장성 및 공정성에 적응할 수 있음을 보여준다. 또한 제안하는 기법이 부하가 증가함에 따라 성능저하를 완화시킴을 확인하였다.
%\vspace*{0.2cm}
%
%\noindent Keywords: Congestion Control, Lotka-Volterra, Dragonfly Algorithm, Industrial Wireless Sensor Networks, Industrial Internet of Things.\\ 

\newpage