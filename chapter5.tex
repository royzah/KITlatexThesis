%-----------------------------------------------------------------------------%
\chapter{Conclusions and Future Works}
%-----------------------------------------------------------------------------%



In this thesis, a hybrid bio-inspired algorithm was proposed for congestion control in large-scale IIoT. First, the C-LV model-based scheme is applied to avoid congestion in WSNs and to maintain fairness among sensor nodes. In addition, the DA is applied to enhances the C-LV scheme by optimizing the parameter for minimizing end-to-end delay and to help the system adapt to changes in the number of sensor nodes.

The results verify that this scheme avoids congestion and maintains fairness for each sensor node. Additionally, each sensor node transmits at a nearly optimal rate with any given number of sensor nodes. The DA algorithm enhances the adaptiveness of the system by adapting rapidly. The simulation results show the proposed scheme works with any topology. In addition, this proposed scheme can be modified therefore some node can be prioritized that is critical in IIoT. 

From a practical perspective, C-LV does not cost much computational power because it work as network framework. The one that cost mild computational power is DA. However, the cost is paid off with such performance. In addition, the scheme works well because DA adapts to changes rapidly as demonstrated by the results. However, the proposed scheme takes enormous computational power when number of node is greater than thousands. For that case, hybrid between decentralized and centralized network based on Software Defined Network (SDN) and network function visualization (NFV) should be considered.

In future work, the proposed method will be evaluated in a real industrial system. In addition, congestion control with another bio-inspired algorithm will be investigated for comparison. 