\begin{center}      
	\vspace*{0.5 cm}
	\begin{spacing}{1.6}
		{\fontsize{22}{30}\selectfont Bio-Inspired Congestion Control Mechanism in Industrial Internet of Things}
	\end{spacing}
	\vspace{1 cm}
	{\fontsize{14}{20}\selectfont Muhammad Royyan}
	
	\vspace{1 cm}
	\begin{spacing}{1.6}
		{\fontsize{14}{20}\selectfont Department of IT Convergence Engineering} \\ {\fontsize{14}{20}\selectfont Graduate School} \\ {\fontsize{14}{20}\selectfont Kumoh National Institute of Technology}
	\end{spacing}
	
	\vspace{1.5 cm}
	{\fontsize{14}{20}\selectfont Abstract}
	
	\vspace{1 cm}
\end{center}


Congestion in a network is determined by the resource constraints and the number of deployed sensor nodes. Congestion can significantly degrade the quality of services (QoS) regarding throughput and end-to-end delay. In this thesis, a hybrid bio-inspired algorithm is proposed for congestion control in large-scale Industrial IoT that necessitate controlled performance with graceful degradation. Each algorithm overcomes each other drawback. First, a competitive Lotka-Volterra (C-LV) model to avoid congestion is employed by regulating the rate of each traffic flow, while fairness among sensor nodes is maintained. However, parameters of C-LV need to be optimized. Thus, dragonfly algorithm (DA) is employed to enhance C-LV by optimizing the parameter for minimizing end-to-end delay. DA makes this scheme adaptive to change and get optimized parameter when there is change number of node due to new installation or faulty node. Performance evaluations verify that the proposed scheme improves the QoS in IIoT environment and achieves adaptability to changing traffic loads, scalability and fairness among flows while providing graceful performance degradation as load increases.
%\vspace*{0.2cm}
%
%\noindent Keywords: Congestion Control, Lotka-Volterra, Dragonfly Algorithm, Industrial Wireless Sensor Networks, Industrial Internet of Things.\\ 

\newpage